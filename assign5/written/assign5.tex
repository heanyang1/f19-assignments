\documentclass[11pt]{article}

%%% PACKAGES

\usepackage{fullpage}  % 1 page margins
\usepackage{latexsym}  % extra symbols, e.g. \leadsto
\usepackage{verbatim}  % verbatim mode
\usepackage{proof}     % proof mode
\usepackage{amsthm,amssymb,amsmath}  % various math symbols
\usepackage{color}     % color control
\usepackage{etoolbox}  % misc utilities
\usepackage{graphics}  % images
\usepackage{mathpartir}% inference rules
\usepackage{hyperref}  % hyperlinks
\usepackage{titlesec}  % title/section controls
\usepackage{minted}    % code blocks
% \usepackage{tocstyle}  % table of contents styling
\usepackage[hang,flushmargin]{footmisc} % No indent footnotes
\usepackage{parskip}   % no parindent
\usepackage{tikz}      % drawing figures
\usepackage[T1]{fontenc}
%%% COMMANDS

% Typography and symbols
\newcommand{\msf}[1]{\mathsf{#1}}
\newcommand{\ctx}{\Gamma}
\newcommand{\qamp}{&\quad}
\newcommand{\qqamp}{&&\quad}
\newcommand{\Coloneqq}{::=}
\newcommand{\proves}{\vdash}
\newcommand{\str}[1]{{#1}^{*}}
\newcommand{\eps}{\varepsilon}
\newcommand{\brc}[1]{\{{#1}\}}
\newcommand{\binopm}[2]{{#1}~\bar{\oplus}~{#2}}
\newcommand{\aequiv}{\equiv_\alpha}

% Untyped lambda calculus
\newcommand{\fun}[2]{\lambda ~ {#1} ~ . ~ {#2}}
\newcommand{\app}[2]{{#1} ~ {#2}}
\newcommand{\fix}[1]{\msf{fix}~{#1}}

% Typed lambda calculus - expressions
\newcommand{\funt}[3]{\lambda ~ ({#1} : {#2}) ~ . ~ {#3}}
\newcommand{\ift}[3]{\msf{if} ~ {#1} ~ \msf{then} ~ {#2} ~ \msf{else} ~ {#3}}
\newcommand{\rec}[5]{\msf{rec}(#1; ~ #2.#3.#4)(#5)}
\newcommand{\lett}[4]{\msf{let} ~ \hasType{#1}{#2} = {#3} ~ \msf{in} ~ {#4}}
\newcommand{\falset}{\msf{false}}
\newcommand{\truet}{\msf{true}}
\newcommand{\case}[5]{\msf{case} ~ {#1} ~ \{ L({#2}) \to {#3} \mid R({#4}) \to {#5} \}}
\newcommand{\pair}[2]{({#1},{#2})}
\newcommand{\proj}[2]{{#1} . {#2}}
\newcommand{\inj}[3]{\msf{inj} ~ {#1} = {#2} ~ \msf{as} ~ {#3}}
\newcommand{\letv}[3]{\msf{let} ~ {#1} = {#2} ~ \msf{in} ~ {#3}}
\newcommand{\fold}[2]{\msf{fold}~{#1}~\msf{as}~{#2}}
\newcommand{\unfold}[1]{\msf{unfold}~{#1}}
\newcommand{\poly}[2]{\Lambda~{#1}~.~{#2}}
\newcommand{\polyapp}[2]{{#1}~[{#2}]}

% Typed lambda calculus - types
\newcommand{\tnum}{\msf{number}}
\newcommand{\tstr}{\msf{string}}
\newcommand{\tint}{\msf{int}}
\newcommand{\tbool}{\msf{bool}}
\newcommand{\tfun}[2]{{#1} \rightarrow {#2}}
\newcommand{\tprod}[2]{{#1} \times {#2}}
\newcommand{\tsum}[2]{{#1} + {#2}}
\newcommand{\trec}[2]{\mu~{#1}~.~{#2}}
\newcommand{\tvoid}{\msf{void}}
\newcommand{\tunit}{\msf{unit}}
\newcommand{\tpoly}[2]{\forall~{#1}~.~{#2}}

% WebAssembly
\newcommand{\wconst}[1]{\msf{i32.const}~{#1}}
\newcommand{\wbinop}[1]{\msf{i32}.{#1}}
\newcommand{\wgetlocal}[1]{\msf{get\_local}~{#1}}
\newcommand{\wsetlocal}[1]{\msf{set\_local}~{#1}}
\newcommand{\wgetglobal}[1]{\msf{get\_global}~{#1}}
\newcommand{\wsetglobal}[1]{\msf{set\_global}~{#1}}
\newcommand{\wload}{\msf{i32.load}}
\newcommand{\wstore}{\msf{i32.store}}
\newcommand{\wsize}{\msf{memory.size}}
\newcommand{\wgrow}{\msf{memory.grow}}
\newcommand{\wunreachable}{\msf{unreachable}}
\newcommand{\wblock}[1]{\msf{block}~{#1}}
\newcommand{\wloop}[1]{\msf{loop}~{#1}}
\newcommand{\wbr}[1]{\msf{br}~{#1}}
\newcommand{\wbrif}[1]{\msf{br\_if}~{#1}}
\newcommand{\wreturn}{\msf{return}}
\newcommand{\wcall}[1]{\msf{call}~{#1}}
\newcommand{\wlabel}[2]{\msf{label}~\{#1\}~{#2}}
\newcommand{\wframe}[2]{\msf{frame}~({#1}, {#2})}
\newcommand{\wtrapping}{\msf{trapping}}
\newcommand{\wbreaking}[1]{\msf{breaking}~{#1}}
\newcommand{\wreturning}[1]{\msf{returning}~{#1}}
\newcommand{\wconfig}[5]{\{\msf{module}{:}~{#1};~\msf{mem}{:}~{#2};~\msf{locals}{:}~{#3};~\msf{stack}{:}~{#4};~\msf{instrs}{:}~{#5}\}}
\newcommand{\wfunc}[4]{\{\msf{params}{:}~{#1};~\msf{locals}{:}~{#2};~\msf{return}~{#3};~\msf{body}{:}~{#4}\}}
\newcommand{\wmodule}[1]{\{\msf{funcs}{:}~{#1}\}}
\newcommand{\wcg}{\msf{globals}}
\newcommand{\wcf}{\msf{funcs}}
\newcommand{\wci}{\msf{instrs}}
\newcommand{\wcs}{\msf{stack}}
\newcommand{\wcl}{\msf{locals}}
\newcommand{\wcm}{\msf{mem}}
\newcommand{\wcmod}{\msf{module}}
\newcommand{\wsteps}[2]{\steps{\brc{#1}}{\brc{#2}}}
\newcommand{\with}{\underline{\msf{with}}}
\newcommand{\wvalid}[2]{{#1} \vdash {#2}~\msf{valid}}
\newcommand{\wif}[2]{\msf{if}~{#1}~{\msf{else}}~{#2}}
\newcommand{\wfor}[4]{\msf{for}~(\msf{init}~{#1})~(\msf{cond}~{#2})~(\msf{post}~{#3})~{#4}}
% assign4.3 custom
\newcommand{\wtry}[2]{\msf{try}~{#1}~\msf{catch}~{#2}}
\newcommand{\wraise}{\msf{raise}}
\newcommand{\wraising}[1]{\msf{raising}~{#1}}

% Inference rules
%\newcommand{\inferrule}[3][]{\cfrac{#2}{#3}\;{#1}}
\newcommand{\ir}[3]{\inferrule*[right=\text{(#1)}]{#2}{#3}}
\newcommand{\s}{\hspace{1em}}
\newcommand{\nl}{\\[2em]}
\newcommand{\steps}[2]{#1 \boldsymbol{\mapsto} #2}
\newcommand{\evals}[2]{#1 \boldsymbol{\overset{*}{\mapsto}} #2}
\newcommand{\subst}[3]{[#1 \rightarrow #2] ~ #3}
\newcommand{\dynJ}[2]{#1 \proves #2}
\newcommand{\dynJC}[1]{\dynJ{\ctx}{#1}}
\newcommand{\typeJ}[3]{#1 \proves \hasType{#2}{#3}}
\newcommand{\typeJC}[2]{\typeJ{\ctx}{#1}{#2}}
\newcommand{\hasType}[2]{#1 : #2}
\newcommand{\val}[1]{#1~\msf{val}}
\newcommand{\num}[1]{\msf{Int}(#1)}
\newcommand{\err}[1]{#1~\msf{err}}
\newcommand{\trans}[2]{#1 \leadsto #2}
\newcommand{\size}[1]{|#1|}

\newcommand{\hwtitle}[2]{\begin{center}{\Large #1} \\[0.5em] {\large #2}\end{center}\vspace{1em}}
\newcommand{\toc}{{\hypersetup{hidelinks}\tableofcontents}}
\newcommand{\problem}[1]{\section*{#1}}

%%% CONFIG

% Spacing around title/sections
\titlelabel{\thetitle.\quad}
\titlespacing*{\section}{0pt}{10pt}{0pt}
\titlespacing*{\subsection}{0pt}{10pt}{0pt}


% Show color on hyperlinks
\hypersetup{colorlinks=true}

% Stylize code blocks
\usemintedstyle{xcode}
% TODO(wcrichto): remove line of spacing beneath code blocks
\newcommand{\nm}[2]{
  \newminted{#1}{#2}
  \newmint{#1}{#2}
  \newmintinline{#1}{#2}}
\nm{lua}{}
\nm{ocaml}{}
\nm{rust}{}
\nm{prolog}{}

\newcommand{\ml}[1]{\ocamlinline|#1|}

% \usetocstyle{standard}


%%%% Useful syntax commands:

% \wci - \mathsf{instrs}
% \wcs - \mathsf{stack}
% \wcm - \mathsf{memory}
% \wcl - \mathsf{locals}

% For example, you can define a configuration as:
% \{\wci{:}~e^*; ~\wcs{:}~n_\wcs^*; ~\wcm{:}~n_\wcm^*; ~\wcl{:}~n_\wcl^*\}

% Each WebAssembly instruction, including the new ones, has a corresponding macro:
% \wconst{n}
% \wbinop{\oplus}
% \wblock{e^*}
% \wloop{e^*}
% \wbr{i}
% \wbrif{i}
% \wlabel{e_\msf{cont}^*}{(n^*; ~e_\msf{body}^*)}
% \wcall{i}
% \wif{e_\msf{then}^*}{e_\msf{else}^*}
% \wfor{e_\msf{init}^*}{e_\msf{cond}^*}{e_\msf{post}^*}{e_\msf{body}^*}
% \wtry{e_\msf{try}^*}{e_\wraise^*}
% \wraise

\begin{document}

\hwtitle
  {Assignment 5}
  {Anyang He (heanyang1)} %% REPLACE THIS WITH YOUR NAME/ID

The T-Call and T-Return semantics that have problem displaying in \href{https://stanford-cs242.github.io/f19/lectures/05-2-control-flow#static-semantics-1}{the lecture note} probably look like:
\begin{mathpar}
\ir{T-Call}
  {f = \ctx.\msf{funcs}[i]}
  {\typeJC{\wcall{i}}{\tfun{f.\msf{params}}{f.\msf{return}}}}

\ir{T-Return}
  {\ctx.\msf{curfunc}.\msf{return} = \tau_\msf{ret}}
  {\typeJC{\wreturn}{\tfun{\tau_1^*, \tau_\msf{ret}}{\tau_2^*}}}
\end{mathpar}

\problem{Problem 2}

Part 1:

\begin{mathpar}
\ir{T-If}
{\typeJC{e_{\msf{then}}^*}{\tfun{\tau_1^*}{\tau_2^*}} \s \typeJC{e_{\msf{else}}^*}{\tfun{\tau_1^*}{\tau_2^*}}}
{\typeJC{\wif{e_{\msf{then}}^*}{e_{\msf{else}}^*}}{\tfun{\tau_1^*,\msf{i32}}{\tau_2^*}}}

% the configuration may be changed (e.g. memory may be written) when stepping from \(e_\msf{then}^*\) to \(\eps\), so we use the verbose ``with'' notation here
\ir{D-If-True}
{\wsteps
  {C~\with~\wci{:}~e_\msf{then}^*; ~\wcs{:}~n^*}
  {C'~\with~\wci{:}~\eps; ~\wcs{:}~n'^*}\s x\neq 0}
{\wsteps
  {C~\with~\wci{:}~\wif{e_{\msf{then}}^*}{e_{\msf{else}}^*}; ~\wcs{:}~n^*,x}
  {C'~\with~\wci{:}~\eps; ~\wcs{:}~n'^*}}

\ir{D-If-False}
{\wsteps
  {C~\with~\wci{:}~e_\msf{else}^*; ~\wcs{:}~n^*}
  {C'~\with~\wci{:}~\eps; ~\wcs{:}~n'^*}}
{\wsteps
  {C~\with~\wci{:}~\wif{e_{\msf{then}}^*}{e_{\msf{else}}^*}; ~\wcs{:}~n^*,0}
  {C'~\with~\wci{:}~\eps; ~\wcs{:}~n'^*}}

\end{mathpar}

Part 2:

We need to add a binary operation to define \(\msf{for}\) loop:
\begin{alignat*}{1}
  \msf{Binop}~\oplus ::= \qamp \oplus \\
  \mid \qamp \msf{eq} \\
\end{alignat*}

A \(\msf{for}\) loop can be decomposed to a \(\msf{block}\) and a \(\msf{loop}\). Inside the \(\msf{for}\) loop,  \(\msf{br}~0\) means ``continue'',  \(\msf{br}~1\) means ``break'', and  \(\msf{br}~x+2\) where \(x\geq0\) goes to the label \(x\) outside the \(\msf{for}\) loop.
\begin{mathpar}

\ir{T-For}
{
  \typeJC{e_{\msf{init}}^*}{\tfun{\eps}{\eps}} \s
  \typeJ{\{\Gamma~\with~\msf{labels}:\Gamma.\msf{labels}+2\}}{e_{\msf{cond}}^*}{\tfun{\eps}{\msf{i32}}} \\
  \typeJ{\{\Gamma~\with~\msf{labels}:\Gamma.\msf{labels}+2\}}{e_{\msf{post}}^*}{\tfun{\eps}{\eps}} \s
  \typeJ{\{\Gamma~\with~\msf{labels}:\Gamma.\msf{labels}+2\}}{e_{\msf{body}}^*}{\tfun{\eps}{\eps}}}
{\typeJC{\wfor{e_{\msf{init}}^*}{e_{\msf{cond}}^*}{e_{\msf{post}}^*}{e_{\msf{body}}^*}}{\tfun{\eps}{\eps}}}

\ir{D-For}
{\ }
{
  \{\wci{:}~\wfor{e_{\msf{init}}^*}{e_{\msf{cond}}^*}{e_{\msf{post}}^*}{e_{\msf{body}}^*}\} \\
  \mapsto\{\wci{:}~e_{\msf{init}}^*,(\msf{block}~(\msf{loop}~e_{\msf{cond}}^*,\msf{i32.const} 0,\msf{i32.eq},\msf{br\underbar{\ }if}~1,e_{\msf{body}}^*,e_{\msf{post}}^*,\msf{br\underbar{\ }if}~0))\}
}

\end{mathpar}

Part 3:

A \(\wraise\) should have the same type as labels (i.e. \(\tfun{\eps}{\eps}\)) so that it can exit labels without violating preservation, but it should consume an \(\msf{i32}\) as error number. My design choice is to add a administrative instruction that can store the error number while having type \(\tfun{\eps}{\eps}\):
\begin{alignat*}{1}
  \msf{Expression}~e ::= \qamp e \\
  \mid \qamp \msf{unwind}\{n\}
\end{alignat*}
where \(n:\msf{i32}\).
\begin{mathpar}
\ir{T-Try}
  {\typeJC{e^*_\msf{try}}{\tfun{\eps}{\eps}}\s\typeJC{e^*_\wraise}{\tfun{\msf{i32}}{\eps}}}
  {\typeJC{\wtry{e^*_\msf{try}}{e^*_\wraise}}{\tfun{\eps}{\eps}}}

\ir{T-Raise}
  {\ }
  {\typeJC{\wraise}{\tfun{\msf{i32}}{\eps}}}

\ir{T-Unwind}
  {\ }
  {\typeJC{\msf{unwind}\{n\}}{\tfun{\msf{\eps}}{\eps}}}

\ir{D-Try-Finish}
  {\ }
  {\steps{\{\wci{:}~\wtry{\eps}{e^*_\wraise}\}}{\{\}}}

\ir{D-Try-Catch}
  {\ }
  {\steps{\{\wci{:}~\wtry{\msf{unwind}\{n\}}{e^*_\wraise}\}}{\{\wci{:}~\msf{i32.const}~n,e^*_\wraise\}}}

\ir{D-Raise}
  {\ }
  {\steps{\{\wci{:}~\wraise;\wcs{:}~n\}}{\{\wci{:}~\msf{unwind}\{n\}\}}}

\ir{D-Unwind}
  {\ }
  {\steps{\{\wci{:}~\wlabel{\underbar{\ }}{(\underbar{\ };\msf{unwind}\{n\},\underbar{\ })}\}}{\{\wci{:}~\msf{unwind}\{n\}\}}}
  
\ir{D-Unwind-Val}
  {\ }
  {\{\wci{:}~\msf{unwind}\{n\}\}\val{}}
\end{mathpar}

Now our \(\msf{unwind}\{n\}\) needs to exit function frame. The problem is functions can have arbitrary return type, and the return value is not ready when a \(\wraise\) happens. My (inelegant) solution is to add a ``dummy'' value for every type and use it as the return value in such cases:
\begin{alignat*}{1}
  \msf{Expression}~e ::= \qamp e \\
  \mid \qamp \msf{dummy}\{\tau\}
\end{alignat*}
Since we only have \(\msf{i32}\) type and the function can only return one value, so these should be sufficient:
\begin{mathpar}
\ir{T-Dummy}
  {\ }
  {\typeJC{\msf{dummy}\{\tau\}}{\tau}}

\ir{D-Dummy-I32}
  {\ }
  {\steps{\{\wci{:}~\msf{dummy}\{\msf{i32}\}\}}{\{\wcs{:}~42\}}}
\end{mathpar}
Then we can escape functions with any types by placing a \(\msf{dummy}\{\tau\}\) to pass preservation check (both sides have type \(\tfun{\eps}{\tau}\)):
\begin{mathpar}
\ir{D-Unwind-Return}
  {C.\msf{module}.\msf{func}[i].\msf{return}=\tau}
  {
    \steps
      {\{C~\with~\wci{:}~\wframe{i}{\{\wci{:}~\msf{unwind}\{n\},\underbar{\ }\}}\}}
      {\{C~\with~\wci{:}~\msf{unwind}\{n\},\msf{dummy}\{\tau\}\}}
  }
\end{mathpar}

Remark: it doesn't matter which value we use as \(\msf{dummy}\{\tau\}\) because it will never be used.

\problem{Problem 3}

Part 1:

\begin{enumerate}
\item \textbf{Undefined behavior:} No. Our type system is independent on memory configuration (\(C.\msf{mem}\) is not involved in any of the static semantics) and the only difference between \(C\) and \(C'\) is \(C.\msf{mem}\neq C'.\msf{mem}\).
\item \textbf{Private function call:} No. The only way to call function 0 is to issue \((\msf{call}~0)\). Since \(C.\msf{inst}\) does not contain \((\msf{call}~0)\) and \(C'.\msf{inst}=C.\msf{inst}\), \(C'.\msf{inst}\) does not contain \((\msf{call}~0)\) neither.
\end{enumerate}

Part 2:

\begin{enumerate}
\item \textbf{Undefined behavior:} The answer is the same as part 1.
\item \textbf{Private function call:} Here's an example where the attacker can use buffer overflow to call the private function 0:
\begin{verbatim}
(module
  (func $a (param $x i32) (result i32)
    (get_local $x)
    (i32.const 1)
    (i32.add))

  (func $b (param $x i32) (result i32)
    (get_local $x)
    (i32.const 2)
    (i32.add))

  (func $main (result i32)
    (i32.const 2)
    (i32.const 1)
    (i32.const 10)
    (i32.store)
    ;; use the first 40 bytes of the memory as buffer to do something
    (i32.const 10)
    (i32.load)
    (call_indirect (param i32) (result i32)))
)
\end{verbatim}
\end{enumerate}


\end{document}
